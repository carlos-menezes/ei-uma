\documentclass[12pt]{article}

\usepackage[portuges]{babel}
\usepackage[utf8]{inputenc}
\usepackage{amsmath}
\usepackage{indentfirst}
\usepackage{multicol,lipsum}
\usepackage{listings}
\usepackage{bookmark}
\usepackage[margin=1in]{geometry}
\usepackage{enumitem}

% \setcounter{tocdepth}{5}
% \setcounter{secnumdepth}{5}

\usepackage{hyperref}
\hypersetup{
    colorlinks,
    citecolor=black,
    filecolor=black,
    linkcolor=black,
    urlcolor=black
}

\newtheorem{definition}{Definição}[section]
\newtheorem{example}{Exemplo}[section]
\newtheorem{note}{Nota}[section]
\newtheorem{exercises}{Exercícios}[section]

\title{Lógica Computacional\\ \Large{Sebenta de Apoio ao Estudo}}
\author{Carlos Menezes}
\date{27 de agosto 2021}

\begin{document}

\maketitle

\newpage

\tableofcontents

\newpage
\newpage

\section{Lógica Proposicional}

\begin{definition}
    Um argumento é uma estrutura da forma:

    \[
    \begin{array}{ cc }
        & \phi_1, \ldots, \phi_n \\
        \cline{2-2}
        & \psi
    \end{array}
    \]

    onde n $\ge$ 1 e $\phi_1,\ldots{},\phi_n$ (\textbf{premissas}) e $\psi$ (\textbf{conclusão do argumento}) são proposições.
    Um argumento da forma acima diz-se \textbf{válido} se e só se a conclusão $\psi$ for verdadeira sempre que as premissas $\phi_1,\ldots,\phi_n$ forem simultaneamente verdadeiras; diz-se um argumento \textbf{inválido} se e só se as premissas forem simultaneamente verdadeiras e a conclusão falsa.

\end{definition}

\subsection{As Linguagens do Cálculo Proposicional}
O \textbf{alfabeto} da maior parte das linguagens proposicionais que serão consideradas é constituído por:

\begin{itemize}
    \item pelos símbolos $($ e $)$.
    \item por um conjunto numerável de \textit{símbolos proposicionais} denotado por $\{ p_1,p_2, \ldots \}$.
    \item pelo conetivo de negação $\neg$ (leia-se: `não`).
    \item por um conjunto finito e não vazio de \textit{conetivos binários}: $\land$, $\lor$, $\to$, $\leftrightarrow$.
\end{itemize}

As fórmulas de uma linguagem proposicional $L$ são as expressões formadas usando os símbolos do alfabeto de $L$ de acordo com as seguintes regras:

\begin{enumerate}[label=(\roman*)]
    \item um símbolo proposicional é uma fórmula atómica.
    \item se $\phi$ é uma fórmula, então ($\neg \phi$) também o é.
    \item se $\phi$ e $\psi$ são fórmulas e $\circ$ é um dos símbolos de conetivos binários do alfabeto de L, então $(\phi\circ\psi)$ é uma fórmula.
\end{enumerate}

\begin{example}
    Seja $L_{\neg,\land,\lor,\to,\leftrightarrow}$. Então:
\end{example}

\textbf{Exemplo de fórmulas de \textit{L}}: $p$, $(\neg r)$, $(\neg(\neg(\neg q)))$, $(p\land q)$\\

\textbf{Exemplos de expressões que \underline{não} são fórmulas de \textit{L}}: $p\neg$, $p\land q$, $\to (r \lor q)$\\

\begin{definition}
    Alguns parêntesis serão omitidos com base na convenção das seguinte precedências entre os operadores conetivos:

    \begin{enumerate}
        \item $\neg$
        \item $\land$
        \item $\lor$
        \item $\leftrightarrow$, $\to$
    \end{enumerate}
\end{definition}

Desta maneira, apresentam-se algumas abreviações:

\begin{note}\end{note}
\begin{itemize}
    \item $p \land q$ é uma abreviação de $(p \land q)$
    \item $p \to \neg q$ é uma abreviação de $(p\to(\neg q))$
    \item $p \to \neg q \lor r$ é uma abreviação de $(p\to((\neg q) \land r))$
\end{itemize}

\begin{enumerate}[label=\arabic*.]
    \item \textbf{Literal}: fórmula que consiste apenas de um símbolos proposicional. \textbf{e.g.} $p_2, \neg p_2$.
    \item $Form(L)$ representa todas as fórmulas de $L$, mas por abuso de notação usa-se $L$ com o mesmo significado de $Form(L)$.
    \item Os símbolos que fazem parte do alfabeto de uma linguagem $L$ são \textbf{símbolos primitivos} da linguagem. A linguagem pode ser estendida com símbolos \textbf{não primitivos}.
\end{enumerate}

\subsection{Uma aplicação do princípio de indução estrutural na lógica proposicional}

Seja $L_{\neg,\circ_1,\ldots,\circ_n}$ uma linguagem proposicional com $n$ conetivos binários ($\circ_1,\ldots,\circ_n$). Para provar que toda a fórmula de $L$ satisfaz uma propriedade $Q$ (i.e. $\forall \psi \in L : Q(\psi)$), basta provar:

\begin{itemize}
    \item \textbf{Base:}
    \begin{description}
        \item Toda a fórmula satisfaz $Q$, i.e. $Q(\psi_i)$
    \end{description}
    \item \textbf{Passo de indução:}
    \begin{enumerate}[label=(\roman*)]
        \item \textbf{Seja $\psi$ arbitrário.} Se se verifica $Q(\psi)$, também se verifica $Q(\neg\psi)$.
        \item \textbf{Seja $\circ_1$ um conetivo binário arbitrário ($i \in \{1,\ldots,n\}$).} Se se verifica $Q(\psi_1)$ e $Q(\psi_2)$, então verifica-se $Q(\psi_1 \circ_i \psi_2)$.
    \end{enumerate}
\end{itemize}

\subsection{Semântica do Cálculo Proposicional}
Se uma proposição é verdadeira, diz-se que tem o valor lógico $1$ (ou $V$ ou $T$); caso contrário, se é falsa, diz-se que tem o valor lógico $0$ (ou $F$).

\begin{definition}
    Uma valoração é uma aplicação

    \[
        v:\{p_1,p_2,\ldots\}\to\{0,1\}
    \]

    que indica o valor de verdade que um dado símbolo proposicional assume para a proposição que ele denota.
\end{definition}

A cada conetivo `$\circ$` do alfabeto de $L$, é associada uma função de verdade $FV_{\circ}:\{0,1\}^n\to\{0,1\}$, onde $n \in \mathbb{N}$ denota a aridade do conetivo $\circ$.

As funções de verdade associadas aos conetivos $\neg$ e $\land$ definem-se como se segue:

\begin{itemize}
    \item $FV_\neg$ é a aplicação $FV_\neg:\{0,1\}\to\{0,1\}$ definida por $FV_\neg(0)=1$ e $FV_\neg(1)=0$ ou simplesmente pela função $FV_\neg(x)=1-x$.
    \item $FV_\land$ é a aplicação $FV_\land:\{0,1\}^2\to\{0,1\}$ definida por \[
        \begin{cases}
            FV_\land(0,0)=0 \\
            FV_\land(0,1)=0 \\
            FV_\land(1,0)=0 \\
            FV_\land(1,1)=1
        \end{cases}\] ou simplesmente pela função $FV_\land(x,y)=x\times y$.
\end{itemize}

\begin{exercises}
    Apresente a definição de cada uma das funções de verdade $FV_\lor$, $FV_\to$ e $FV_\leftrightarrow$ associadas aos conetivos $\lor$, $\to$ e $\leftrightarrow$, respetivamente.

    \begin{itemize}
        \item $FV_\lor$ é a aplicação $FV_\lor:\{0,1\}^2\to\{0,1\}$ definida por \[
            \begin{cases}
                FV_\land(0,0)=0 \\
                FV_\land(0,1)=1 \\
                FV_\land(1,0)=1 \\
                FV_\land(1,1)=1
            \end{cases}\] ou simplesmente pela função $FV_\lor(x,y)=max(x, y)$.
        \item $FV_\to$ é a aplicação $FV_{\to}:\{0,1\}^2\to\{0,1\}$ definida por \[
            \begin{cases}
                FV_\land(0,0)=1 \\
                FV_\land(0,1)=1 \\
                FV_\land(1,0)=0 \\
                FV_\land(1,1)=1
            \end{cases}\].
        \item $FV_\leftrightarrow$ é a aplicação $FV_{\leftrightarrow}:\{0,1\}^2\to\{0,1\}$ definida por \[
            \begin{cases}
                FV_\land(0,0)=1 \\
                FV_\land(0,1)=0 \\
                FV_\land(1,0)=0 \\
                FV_\land(1,1)=1
            \end{cases}\].
    \end{itemize}
\end{exercises}

Seja $v$ uma valoração dos símbolos proposicionais e $FV_\circ$ as funções e verdade para cada conetivo $\circ$ do alfabeto de $L$. O valor de verdade das fórmulas não atómicas (de $L$) define-se recursivamente como se segue:

\begin{enumerate}[label=(\roman*)]
    \item $v(\neg\psi)=FV_\neg(v(\psi))$;
    \item Se $\circ$ é um conetivo binário da linguagem $L$, $v(\psi_1 \circ \psi_2)=FV_\circ(v(\psi_1),v(\psi_2))$.
\end{enumerate}

Considerando uma valoração $v$ tal que $v(p_1)=1$ $v(p_2)=0$, o valor lógico da fórmula $p_1\land p_2 \to \negp_1$ pode ser calculado recorrendo às funções de verdade referidas acima:

    \begin{align*}
        v(p_1\land p_2 \to \negp_1)& =FV_{\to}(v(p_1\land p_2),v(\neg p_1)) \\
        & =FV_{\to}(FV_\land(v(p_1), v(p_2)), FV_\neg(v(p_1)))\\
        & =FV_{\to}(FV_\land(1,0),FV_\neg(1)) \\
        & =FV_{\to}(0,0) \\
        & =1
    \end{align*}

\end{document}