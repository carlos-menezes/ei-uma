\section{Lógica Proposicional}

\begin{definition}
    Um argumento é uma estrutura da forma:

    \[
    \begin{array}{ cc }
        & \phi_1, \ldots, \phi_n \\
        \cline{2-2}
        & \psi
    \end{array}
    \]

    onde n $\ge$ 1 e $\phi_1,\ldots{},\phi_n$ (\textbf{premissas}) e $\psi$ (\textbf{conclusão do argumento}) são proposições.
    Um argumento da forma acima diz-se \textbf{válido} se e só se a conclusão $\psi$ for verdadeira sempre que as premissas $\phi_1,\ldots,\phi_n$ forem simultaneamente verdadeiras; diz-se um argumento \textbf{inválido} se e só se as premissas forem simultaneamente verdadeiras e a conclusão falsa.

\end{definition}

\subsection{As Linguagens do Cálculo Proposicional}
O \textbf{alfabeto} da maior parte das linguagens proposicionais que serão consideradas é constituído por:

\begin{itemize}
    \item pelos símbolos $($ e $)$.
    \item por um conjunto numerável de \textit{símbolos proposicionais} denotado por $\{ p_1,p_2, \ldots \}$.
    \item pelo conetivo de negação $\neg$ (leia-se: `não`).
    \item por um conjunto finito e não vazio de \textit{conetivos binários}: $\land$, $\lor$, $\implies$, $\iff$.
\end{itemize}

As fórmulas de uma linguagem proposicional $L$ são as expressões formadas usando os símbolos do alfabeto de $L$ de acordo com as seguintes regras:

\begin{enumerate}[label=(\roman*)]
    \item um símbolo proposicional é uma fórmula atómica.
    \item se $\phi$ é uma fórmula, então ($\neg \phi$) também o é.
    \item se $\phi$ e $\psi$ são fórmulas e $\circ$ é um dos símbolos de conetivos binários do alfabeto de L, então $(\phi\circ\psi)$ é uma fórmula.
\end{enumerate}

\begin{example}
    Seja $L_{\neg,\land,\lor,\implies,\iff}$. Então:
\end{example}

\textbf{Exemplo de fórmulas de \textit{L}}: $p$, $(\neg r)$, $(\neg(\neg(\neg q)))$, $(p\land q)$\\

\textbf{Exemplos de expressões que \underline{não} são fórmulas de \textit{L}}: $p\neg$, $p\land q$, $\implies (r \lor q)$\\

\begin{definition}
    Alguns parêntesis serão omitidos com base na convenção das seguinte precedências entre os operadores conetivos:

    \begin{enumerate}
        \item $\neg$
        \item $\land$
        \item $\lor$
        \item $\iff$, $\implies$
    \end{enumerate}
\end{definition}

Desta maneira, apresentam-se algumas abreviações:

\begin{note}\end{note}
\begin{itemize}
    \item $p \land q$ é uma abreviação de $(p \land q)$
    \item $p \implies \neg q$ é uma abreviação de $(p\implies(\neg q))$
    \item $p\implies \neg q \lor r$ é uma abreviação de $(p\implieds((\neg q) \land r))$
\end{itemize}

\begin{enumerate}[label=\arabic*.]
    \item \textbf{Literal}: fórmula que consiste apenas de um símbolos proposicional. \textbf{e.g.} $p_2, \neg p_2$.
    \item $Form(L)$ representa todas as fórmulas de $L$, mas por abuso de notação usa-se $L$ com o mesmo significado de $Form(L)$.
    \item Os símbolos que fazem parte do alfabeto de uma linguagem $L$ são \textbf{símbolos primitivos} da linguagem. A linguagem pode ser estendida com símbolos \textbf{não primitivos}.
\end{enumerate}